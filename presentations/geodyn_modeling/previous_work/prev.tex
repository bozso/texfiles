\documentclass{beamer}
\usepackage[english]{babel}
\usepackage[utf8]{inputenc}
\usepackage[T1]{fontenc}
\usepackage{graphicx, tikz, multicol, wrapfig}
\usepackage[framemethod=TikZ]{mdframed}


\mode<presentation>
{
    \usetheme{Boadilla}
    \usecolortheme{default}
    \usefonttheme{default}
    \setbeamertemplate{navigation symbols}{}
    \setbeamertemplate{caption}[numbered]
} 


\title[GGI, Sopron 2019]{Previous studies regarding the numercial simulation of
                         the Vrancea subduction}
\author[István Bozsó]{István Bozsó}
\institute[MTA CSFK GGI]{MTA CSFK Geodetic and Geophysical Institute}
\date{2019.10.00.}

\def\rar{\rightarrow}
\def\lar{\leftarrow}


\newcommand\inc[1] {
    \includegraphics[width=\textwidth]{#1}
}

\newcommand{\ffig}[1]{
    \begin{mdframed}[linecolor=blue!65!white, linewidth=1.75pt, roundcorner=0.75pt,
                     innerrightmargin=0pt, innerleftmargin=0pt,
                     innertopmargin=0pt, innerbottommargin=0pt,
                     backgroundcolor=white, frametitle={}, align=center]
        \includegraphics[width=1.0\textwidth]{#1}
    \end{mdframed}
}


\newcommand\sepframe[1] {
    \begin{frame}
        \begin{center}
            \Huge \color{blue!55!black}
            #1
        \end{center}
    \end{frame}
}


\graphicspath{{../../../images/}}


\begin{document}

\begin{frame}
    \titlepage
    \begin{center}
        \begin{minipage}[c]{0.3\textwidth}
            \includegraphics[width=0.75\textwidth]{ggi_logo.png}
        \end{minipage}
    \end{center}
\end{frame}


\def\ft{Introduction}

\begin{frame}{\ft}
    \begin{itemize}
        \item Vrancea seismic zone; located in Romania, at the eastern edge
        of the south-eastern Carpathians
        \item intra-continental, intermediate depth seismicity
        \item interesting topic: link between shallow and deep structures and
            processes; crustal and mantle dynamics
        \item information from geology, geophysics, geodesy:
        
        \pause
        
        \begin{itemize}
            \item post-collisional to recent deformations,
            \item syn- to post collisional magmatism,
            \item orogenic exhumation along the East and South Carpathians,
            \item seismic studies; reflection, refraction, body and surface wave
            tomography,
            \item geoelectric studies, density/gravity and thermal modelling,
            \item qualitative and quantitative geodynamic models developed for Vrancea,
            \item horizontal and vertical movements.
        \end{itemize}
    
    \pause
    
    \item earthquake simulation, seismic hazard, and earthquake forecasting
    models $\rightarrow$ link between deep geodynamic processes and their
    manifestation on the surface
    
    \end{itemize}
\end{frame}

\begin{frame}{\ft}
    \begin{itemize}
        \item origin of the high-velocity body revealed by seismic tomography,
        \item lithospheric scale mechanism driving the Miocene subsidence
        of the Transylvania basin,
        \item sub-crustal structure between 40 and 70 \(\kilo\meter\)
        \item contemporary regional horizontal and vertical movements
        \item comprehensive seismic hazard assessment in the region

    \end{itemize}
\end{frame}

\end{document}
