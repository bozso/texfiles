\documentclass[aspectratio=169]{beamer}

\mode<presentation>
{
    \usetheme{Boadilla}
    \usecolortheme{default}
    \usefonttheme{default}
    \setbeamertemplate{navigation symbols}{}
    \setbeamertemplate{caption}[numbered]
} 

\usepackage[magyar]{babel}
\usepackage[utf8]{inputenc}
\usepackage[T1]{fontenc}
\usepackage{graphicx, tikz}
\usepackage[framemethod=TikZ]{mdframed}

\newcommand\annotatedFigureBoxCustom[8]{
    \draw[#5,ultra thick,rounded corners] (#1) rectangle (#2);
    \node at (#4) [fill=#6,thick,shape=circle,draw=#7,
                   inner sep=2pt,font=\sffamily,text=#8]
                   {\textbf{#3}};
}


% {from}{to}{text}{color}
\newcommand\anbox[4]{\annotatedFigureBoxCustom{#1}{#2}{#3}{#1}{#4}{white}{black}{black}}


\newcommand\antext[5]{
    \draw[#4, fill=#5, ultra thick, rounded corners] (#1) rectangle (#2)
    node[pos=0.5, font=\sffamily, text=black] {\textbf{#3}};
}


\newenvironment{anfig}[1]
{
    \begin{mdframed}[linecolor=blue!65!white, linewidth=2pt, roundcorner=0.75pt,
                     innerrightmargin=0.05pt, innerleftmargin=0.05pt,
                     innertopmargin=0.05pt, innerbottommargin=0.05pt, frametitle={}]
    \begin{tikzpicture}
    \node[anchor=south west,inner sep=0] (image) at (0,0)
    {\includegraphics[width=\textwidth]{#1}};
    \begin{scope}[x={(image.south east)},y={(image.north west)}]
}
{
    \end{scope}
    \end{tikzpicture}
    \end{mdframed}
}

\newcommand{\ffig}[1]{
    \begin{mdframed}[linecolor=blue!65!white, linewidth=1.75pt, roundcorner=0.75pt,
                     innerrightmargin=0pt, innerleftmargin=0pt,
                     innertopmargin=0pt, innerbottommargin=0pt,
                     backgroundcolor=white, frametitle={}, align=center]
        \includegraphics[width=1.0\textwidth]{#1}
    \end{mdframed}
}


\newcommand\sepframe[1] {
    \begin{frame}
        \begin{center}
            \Huge \color{blue!55!black}
            #1
        \end{center}
    \end{frame}
}

\newcommand\bench[2]{
    \begin{minipage}[t]{0.45\textwidth}
        \centering
        \includegraphics[width=\textwidth]{#1}
        \vspace{7.5pt}
        
        LOS deformáció.
    \end{minipage}
    \hspace{5pt}
    \begin{minipage}[t]{0.45\textwidth}
        \centering
        \includegraphics[width=\textwidth]{#2}
        \vspace{7.5pt}
        
        Kálmán-szűréssel integrált GNSS és InSAR mérések.
    \end{minipage}
}

\def\boxcolor{black}
\def\fillcolor{white}
\def\refcolor{orange}


\def\hun{
    \begin{anfig}{hungary_networks_raw.png}
        \anbox{0.27, 0.3}{0.33, 0.4}{A}{\boxcolor}
        \anbox{0.5, 0.4}{0.5525, 0.525}{B}{\boxcolor}
        \anbox{0.465, 0.05}{0.535, 0.15}{C}{\boxcolor}
    \end{anfig}
}

\def\erdely{
    \begin{anfig}{transylvania_networks_raw.png}
        \anbox{0.4075, 0.55}{0.4875, 0.67}{A}{\boxcolor}
        \anbox{0.55, 0.37}{0.63, 0.49}{B}{\boxcolor}
        \anbox{0.75, 0.08}{0.9, 0.4}{C}{\boxcolor}
    \end{anfig}
}

\def\dszekcso{
    \begin{anfig}{dszekcso_network_raw.png}
        \antext{0.05, 0.57}{0.25, 0.64}{IB1}{\refcolor}{\fillcolor}
        \antext{0.3, 0.67}{0.5, 0.74}{IB2}{\boxcolor}{\fillcolor}
        \antext{0.225, 0.47}{0.425, 0.54}{IB3}{\boxcolor}{\fillcolor}
        \antext{0.32, 0.25}{0.52, 0.32}{IB4}{\boxcolor}{\fillcolor}
    \end{anfig}
}

\def\kulcs{
    \begin{anfig}{kulcs_network_raw.png}
        \antext{0.125, 0.875}{0.225, 0.975}{A1}{\boxcolor}{\fillcolor}
        \antext{0.51, 0.45}{0.61, 0.55}{A2}{\boxcolor}{\fillcolor}
        \antext{0.825, 0.17}{0.925, 0.27}{A3}{\boxcolor}{\fillcolor}
        \antext{0.79, 0.04}{0.89, 0.14}{A4}{\boxcolor}{\fillcolor}
        \antext{0.46, 0.17}{0.56, 0.27}{AR}{\refcolor}{\fillcolor}
        \antext{0.475, 0.7}{0.625, 0.8}{Duna}{\boxcolor}{\fillcolor}
    \end{anfig}
}

\def\fonyod{
    \begin{anfig}{fonyod_network_raw.png}
        \antext{0.25, 0.78}{0.45, 0.86}{Balaton}{\boxcolor}{\fillcolor}
        \antext{0.725, 0.77}{0.825, 0.87}{KV}{\boxcolor}{\fillcolor}
        \antext{0.725, 0.16}{0.825, 0.26}{PH}{\refcolor}{\fillcolor}
        \antext{0.29, 0.0575}{0.39, 0.1575}{RE}{\boxcolor}{\fillcolor}
    \end{anfig}
}


\def\csomad{
    \begin{anfig}{ciomadul_network_raw.png}
        \antext{0.3, 0.29}{0.42, 0.375}{C1}{\boxcolor}{\fillcolor}
        \antext{0.21, 0.1}{0.33, 0.18}{C2}{\refcolor}{\fillcolor}
        \antext{0.665, 0.2}{0.775, 0.28}{C3}{\boxcolor}{\fillcolor}
        \antext{0.49, 0.565}{0.6, 0.645}{C4}{\boxcolor}{\fillcolor}
        \antext{0.64, 0.82}{0.76, 0.9}{C5}{\refcolor}{\fillcolor}
    \end{anfig}
}



\graphicspath{{../../images/}}


\title[Geomatika Szeminárium 2018]{Lassú felszíni deformációs folyamatok monitorozása sarokreflektorok és Sentinel-1 adatok felhasználásával magyarországi teszt területeken}
\author[Bozsó et al.]{Bozsó István, Szűcs Eszter, Bányai László, Wesztergom Viktor}
\institute[MTA CSFK GGI]{MTA CSFK Geodéziai és Geofizikai Intézet}
\date{2018.11.08.}


\begin{document}

\begin{frame}
    \titlepage
    \begin{center}
        \begin{minipage}[c]{0.25\textwidth}
            \includegraphics[width=0.75\textwidth]{ggi_logo.png}
        \end{minipage}
        \hspace{30pt}
        \begin{minipage}[c]{0.25\textwidth}
            \includegraphics[width=0.75\textwidth]{esa_logo.eps}
        \end{minipage}
    \end{center}
\end{frame}

\begin{frame}{Áttekintés}
    \tableofcontents
\end{frame}

\section{Bevezetés}

\begin{frame}{Bevezetés}
    
\end{frame}





\section{A reflektor hálózatok első eredményei}

\sepframe{Magyarországi reflektor hálózatok és az első eredmények}

\begin{frame}{Magyarországi reflektor hálózatok}
    \centering
    \begin{minipage}[c]{0.7\textwidth}
        \hun
        \begin{center}
            \textbf{A}: Fonyód,
            \textbf{B}: Kulcs,
            \textbf{C}: Dunaszekcső
        \end{center}
    \end{minipage}
\end{frame}


\begin{frame}{Magyarországi reflektor hálózatok}
    \begin{minipage}[c]{0.365\textwidth}
        \centering
        \dszekcso

        Dunaszekcsői hálózat
    \end{minipage}
    \hspace{10pt}
    \begin{minipage}[c]{0.58\textwidth}
        \centering
        \kulcs

        Kulcsi hálózat
    \end{minipage}
\end{frame}


% ***************
% * Dunaszekcső *
% ***************


\begin{frame}{Dunaszekcsői hálózat IB2-IB1}
    \bench{IB2-IB1_los.png}{IB2-IB1_kalman.png}
\end{frame}

\begin{frame}{Dunaszekcsői hálózat IB3-IB1}
    \bench{IB3-IB1_los.png}{IB3-IB1_kalman.png}
\end{frame}

\begin{frame}{Dunaszekcsői hálózat IB4-IB1}
    \bench{IB4-IB1_los.png}{IB4-IB1_kalman.png}
\end{frame}


\begin{frame}{Magyarországi reflektor hálózatok}
    \begin{minipage}[c]{0.365\textwidth}
        \centering
        \dszekcso

        Dunaszekcsői hálózat
    \end{minipage}
    \hspace{10pt}
    \begin{minipage}[c]{0.58\textwidth}
        \centering
        \kulcs

        Kulcsi hálózat
    \end{minipage}
\end{frame}


% *********
% * Kulcs *
% *********


\begin{frame}{Kulcsi hálózat A1-AR}
    \bench{A1-AR_los.png}{A1-AR_kalman.png}
\end{frame}


\begin{frame}{Kulcsi hálózat A3-AR}
    \bench{A3-AR_los.png}{A3-AR_kalman.png}
\end{frame}


\begin{frame}{Magyarországi reflektor hálózatok}
    \begin{minipage}[c]{0.6\textwidth}
        \centering
        \fonyod

        Fonyódi hálózat
    \end{minipage}
    \hspace{5pt}
    \begin{minipage}[c]{0.3\textwidth}
            \textbf{PH}: Polgármesteri hivatal\\
            \textbf{RE}: Ripka emlékmű\\
            \textbf{KV}: Kripta-villa
    \end{minipage}
\end{frame}


% **********
% * Fonyód *
% **********

\begin{frame}{Fonyód network KV-PH}
    \bench{KV-PH_los.png}{KV-PH_kalman.png}
\end{frame}

\begin{frame}{Fonyód network RE-PH}
    \bench{RE-PH_los.png}{RE-PH_kalman.png}
\end{frame}



\section{Telepített és tervezett erdélyi hálózatok}

\sepframe{Erdélyi reflektor hálózatok}

\begin{frame}{Erdélyi reflektor hálózatok}
    \centering
    \begin{minipage}[c]{0.65\textwidth}
        \erdely
        \begin{center}
            \textbf{A}: Parajdi hálózat, 
            \textbf{B}: Csomádi hálózat, 
            \textbf{C}: Vrancea-zóna
        \end{center}
    \end{minipage}
\end{frame}


\begin{frame}{Parajdi hálózat}
    \begin{minipage}[c]{0.7\textwidth}
        \includegraphics[width=\textwidth]{parajd_3D.png}
    \end{minipage}
    \hspace{5pt}
    \begin{minipage}[c]{0.225\textwidth}
        \begin{itemize}
            \item Közép-Európa legnagyobb sódiapírja $\rightarrow$ sótektonika
            \item vertikális mozgások?
        \end{itemize}
    \end{minipage}
\end{frame}


\begin{frame}{Csomádi hálózat}
    \begin{minipage}[c]{0.475\textwidth}
        \centering
        \csomad
    \end{minipage}
    \hspace{5pt}
    \begin{minipage}[c]{0.45\textwidth}
        \begin{itemize}
            \item magnetotellurikus mérések $\rightarrow$ potenciálisan aktív magmatározó
            \item lehetséges mozgások a Csomád-vulkán környezetében
        \end{itemize}
    \end{minipage}
\end{frame}

\begin{frame}
    \begin{center}
        \Huge \color{blue!55!black}
        Köszönöm a figyelmet!
    \end{center}
\end{frame}

\end{document}
