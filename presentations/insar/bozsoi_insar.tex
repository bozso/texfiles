\documentclass[aspectratio=169]{beamer}
\mode<presentation>
{
    \usetheme{Boadilla}
    \usecolortheme{default}
    \usefonttheme{default}
    \setbeamertemplate{navigation symbols}{}
    \setbeamertemplate{caption}[numbered]
} 

\usepackage[magyar]{babel}
\usepackage[utf8]{inputenc}
\usepackage[T1]{fontenc}
\usepackage{graphicx, tikz, pgf, pgfplots, booktabs, amsmath}
\usepgfplotslibrary{polar}
\usepackage[framemethod=TikZ]{mdframed}
\bibliographystyle{apalike}

\newcommand\annotatedFigureBoxCustom[8]{
    \draw[#5,ultra thick,rounded corners] (#1) rectangle (#2);
    \node at (#4) [fill=#6,thick,shape=circle,draw=#7,
                   inner sep=2pt,font=\sffamily,text=#8]
                   {\textbf{#3}};
}


% {from}{to}{text}{color}
\newcommand\anbox[4]{\annotatedFigureBoxCustom{#1}{#2}{#3}{#1}{#4}{white}{black}{black}}


\newcommand\antext[5]{
    \draw[#4, fill=#5, ultra thick, rounded corners] (#1) rectangle (#2)
    node[pos=0.5, font=\sffamily, text=black] {\textbf{#3}};
}


\newenvironment{anfig}[1]
{
    \begin{mdframed}[linecolor=blue!65!white, linewidth=2pt, roundcorner=0.75pt,
                     innerrightmargin=0.05pt, innerleftmargin=0.05pt,
                     innertopmargin=0.05pt, innerbottommargin=0.05pt, frametitle={}]
    \begin{tikzpicture}
    \node[anchor=south west,inner sep=0] (image) at (0,0)
    {\includegraphics[width=\textwidth]{#1}};
    \begin{scope}[x={(image.south east)},y={(image.north west)}]
}
{
    \end{scope}
    \end{tikzpicture}
    \end{mdframed}
}

\def\rar{\rightarrow}
\def\lar{\leftarrow}


\newcommand\inc[1] {
    \includegraphics[width=\textwidth]{#1}
}

\newcommand{\ffig}[1]{
    \begin{mdframed}[linecolor=blue!65!white, linewidth=1.75pt, roundcorner=0.75pt,
                     innerrightmargin=0pt, innerleftmargin=0pt,
                     innertopmargin=0pt, innerbottommargin=0pt,
                     backgroundcolor=white, frametitle={}, align=center]
        \includegraphics[width=1.0\textwidth]{#1}
    \end{mdframed}
}


\newcommand\sepframe[1] {
    \begin{frame}
        \begin{center}
            \Huge \color{blue!55!black}
            #1
        \end{center}
    \end{frame}
}

\newcommand\bench[2]{
    \begin{minipage}[t]{0.45\textwidth}
        \centering
        \includegraphics[width=\textwidth]{#1}
        \vspace{7.5pt}
        
        LOS deformáció.
    \end{minipage}
    \hspace{5pt}
    \begin{minipage}[t]{0.45\textwidth}
        \centering
        \includegraphics[width=\textwidth]{#2}
        \vspace{7.5pt}
        
        Kálmán-szűréssel integrált GNSS és InSAR mérések.
    \end{minipage}
}

\def\boxcolor{black}
\def\fillcolor{white}
\def\refcolor{orange}


\def\hun{
    \begin{anfig}{hungary_networks_raw.png}
        \anbox{0.27, 0.3}{0.33, 0.4}{A}{\boxcolor}
        \anbox{0.5, 0.4}{0.5525, 0.525}{B}{\boxcolor}
        \anbox{0.465, 0.05}{0.535, 0.15}{C}{\boxcolor}
    \end{anfig}
}

\def\erdely{
    \begin{anfig}{transylvania_networks_raw.png}
        \anbox{0.4075, 0.55}{0.4875, 0.67}{A}{\boxcolor}
        \anbox{0.55, 0.37}{0.63, 0.49}{B}{\boxcolor}
        \anbox{0.75, 0.08}{0.9, 0.4}{C}{\boxcolor}
    \end{anfig}
}

\def\dszekcso{
    \begin{anfig}{dszekcso_network_raw.png}
        \antext{0.05, 0.57}{0.25, 0.64}{IB1}{\refcolor}{\fillcolor}
        \antext{0.3, 0.67}{0.5, 0.74}{IB2}{\boxcolor}{\fillcolor}
        \antext{0.225, 0.47}{0.425, 0.54}{IB3}{\boxcolor}{\fillcolor}
        \antext{0.32, 0.25}{0.52, 0.32}{IB4}{\boxcolor}{\fillcolor}
    \end{anfig}
}

\def\kulcs{
    \begin{anfig}{kulcs_network_raw.png}
        \antext{0.125, 0.875}{0.225, 0.975}{A1}{\boxcolor}{\fillcolor}
        \antext{0.51, 0.45}{0.61, 0.55}{A2}{\boxcolor}{\fillcolor}
        \antext{0.825, 0.17}{0.925, 0.27}{A3}{\boxcolor}{\fillcolor}
        \antext{0.79, 0.04}{0.89, 0.14}{A4}{\boxcolor}{\fillcolor}
        \antext{0.46, 0.17}{0.56, 0.27}{AR}{\refcolor}{\fillcolor}
        \antext{0.475, 0.7}{0.625, 0.8}{Duna}{\boxcolor}{\fillcolor}
    \end{anfig}
}

\def\fonyod{
    \begin{anfig}{fonyod_network_raw.png}
        \antext{0.25, 0.78}{0.45, 0.86}{Balaton}{\boxcolor}{\fillcolor}
        \antext{0.725, 0.77}{0.825, 0.87}{KV}{\boxcolor}{\fillcolor}
        \antext{0.725, 0.16}{0.825, 0.26}{PH}{\refcolor}{\fillcolor}
        \antext{0.29, 0.0575}{0.39, 0.1575}{RE}{\boxcolor}{\fillcolor}
    \end{anfig}
}


\def\csomad{
    \begin{anfig}{ciomadul_network_raw.png}
        \antext{0.3, 0.29}{0.42, 0.375}{C1}{\boxcolor}{\fillcolor}
        \antext{0.21, 0.1}{0.33, 0.18}{C2}{\refcolor}{\fillcolor}
        \antext{0.665, 0.2}{0.775, 0.28}{C3}{\boxcolor}{\fillcolor}
        \antext{0.49, 0.565}{0.6, 0.645}{C4}{\boxcolor}{\fillcolor}
        \antext{0.64, 0.82}{0.76, 0.9}{C5}{\refcolor}{\fillcolor}
    \end{anfig}
}



\graphicspath{{../../images/}}

\title[Műholdas Távérzékelés Labor, 2019/20.I.]{Műholdas Radar Interferometria elmélete és alkalmazásai}
\author[Bozsó István]{Bozsó István}
\institute[ELKH CSFK GGI]{ELKH CSFK Geodéziai és Geofizikai Intézet}
\date{2020.11.13.}


%\includeonly{sections/ts_an}

\begin{document}

\begin{frame}
    \titlepage
    \begin{center}
        \begin{minipage}[c]{0.3\textwidth}
            \includegraphics[width=0.75\textwidth]{logos/ggi.png}
        \end{minipage}
    \end{center}
\end{frame}


\def\ft{Ismétlés - Szintetikus Apertúrájú Radar}

\begin{frame}{\ft}
    \begin{itemize}
        \item kiváló térbeli felbontás, egy csatornán (mikrohullámú tartomány)
        \begin{itemize}
            \item L-sáv: 1-2 GHz; 30-15 cm (ALOS-PALSAR (JAXA))
            \item S-sáv: 2-4 GHz; 15-7.5 cm
            \item C-sáv: 4-8 GHz; 7.5-3.8 cm (Sentinel-1 (ESA), ERS (ESA), Envisat (ESA))
            \item X-sáv: 8-12 GHz; 3.82.5 cm (TerraSAR-X (DLR), TanDEM-X(DLR))
        \end{itemize}
        \item koherens radar: amplitúdó ÉS fázis regisztrálása
        \item alkalmazás amplitúdóra: Freidl Zoli előadása
    \end{itemize}
\end{frame}


\begin{frame}{\ft}
    % \cite{RamonHanssen}
    \begin{figp}{\inc{insar/theory/ramon_SAR.png}}{Radar felvétel geometriája \cite{RamonHanssen}}{0.45}{0.4}
        RAR (Real Aperture Radar) felbontása: azimut / slant range $\approx 5$ km, ground range $\approx 14$ km:
        $$ \subt{W}{a} = \frac{\lambda}{\subt{L}{a}} R $$
        $$ \subt{\Delta}{r} = \frac{c\tau}{2} $$
    \end{figp}
\end{frame}

\def\ft{Range felbontás növelése}

\begin{frame}{\ft}
    \begin{figp}{\inc{insar/theory/chirp_demod.png}}{}{0.4}{0.4}
        Chirp jel kibocsájtása, visszavert jel feldolgozása illesztett szűréssel:
        \begin{align*}
            f(t) &= a t & \text{lineárisan növekvő frekvencia} \\
            \subt{B}{R} &= a \tau & \text{sávszélesség} \\
            \subt{\Delta}{r} &= \frac{c}{2 \subt{B}{R}} \approx 9.6 m & \text{slant range felbontás}
        \end{align*}
        $\approx 25$ m ground range felbontás
    \end{figp}
\end{frame}

\def\ft{Azimut felbontás növelése}

\begin{frame}{\ft}
    \begin{center}
    \begin{minipage}[b]{0.4\textwidth}
        \inc{insar/theory/sar_synthetic_length.png}
        \centering
        
        Egy földfelszíni szórópont leképződése a SAR felvétel készítése során.
    \end{minipage}
    \hspace{10pt}
    \begin{minipage}[b]{0.5\textwidth}
        \inc{insar/theory/chirp_azimuth.png}
        \centering
        
        A földfelszíni szórópont által visszavert chirp jelek.
    \end{minipage}
    \end{center}
\end{frame}


\def\ft{Nyers SAR felvétel fókuszálása}

\begin{frame}{\ft}
    \begin{minic}{0.70}
        \inc{insar/theory/rng_azi_compression.png}
        
        \centering
        Felbontás fókuszálás után: ground range: $\approx 25$ m, azimut: $\approx 10$ m (ERS-1,2)
    \end{minic}
\end{frame}


\def\ft{Ismétlés - Szintetikus Apertúrájú Radar}

\begin{frame}{\ft}
    \begin{minic}{0.80}
        \inc{insar/theory/mw_bands.png}
        
        \centering
        Különböző radarsávban készült felvételek egy területről.
    \end{minic}
\end{frame}


\begin{frame}{\ft}
    % insartut
    \begin{figp}{\inc{insar/theory/sar_speckle.png}}{}{0.45}{0.45}
        \begin{itemize}
            \item amplitúdó ÉS fázis érték pixelenként; komplex számként reprezentálható
            \item egyetlen SAR pixel értéke: $\subt{C}{SAR} = A e^{i\phi}$
            \item $\subt{C}{SAR} = \sum_j c_j$
        \end{itemize}
    \end{figp}
\end{frame}


\begin{frame}{\ft}
    \begin{minic}{0.64}
        %RamonHanssen
        \inc{insar/theory/ramon_shortening.png}
        \centering
        
        Topográfia által okozott geometriai torzulásai a felbontási cellának.
    \end{minic}
\end{frame}


\begin{frame}{\ft}
    \begin{minic}{0.85}
        \inc{insar/theory/mt_merapi.png}
        \centering
        
        Különböző felvételi geometriából készült felvételek a Merapi-hegyről (Indonézia)
    \end{minic}
\end{frame}

\newcommand{\sinPhase}[2] {
    \addplot[smooth, domain=0:360,line width=1.5pt, #2]{sin(x - (#1))};
}

\newcommand{\phase}[2] {
    \addplot[->, >=latex, line width=2.5pt, #2] coordinates { (0,0) (#1,1.5) };
}

\newcommand{\wavesAmpl}[1]{
\begin{tikzpicture}[baseline,scale=#1]
\begin{axis}[
    domain=0.0:360,
    xlabel={$x$ (deg)},
    ylabel={$y = f(x)$},
    enlarge x limits=false,
    legend style={at={(0.5, -0.2)}, anchor=north, legend columns=3},
]
    \sinPhase{0}{blue}
    \sinPhase{90}{red}
    \sinPhase{210}{green}
    \legend{$\sin(x)$, $\sin(x - 90^\circ)$, $\sin(x - 210^\circ)$}
\end{axis}
\end{tikzpicture}
}


\newcommand{\wavesPhase}[1]{
\begin{tikzpicture}[baseline,scale=#1]
\begin{polaraxis} [
    xlabel={Fázis (deg)},
    ytick = \empty,
]
    \phase{0}{blue}
    \phase{90}{red}
    \phase{210}{green}
\end{polaraxis}
\end{tikzpicture}
}


\sepframe{Radarinterferometriai alapok}
\def\ft{Periodukus függvény fázisának tulajdonságai}

\begin{frame}{\ft}
    \begin{center}
    \begin{tabular}{rl}
    \wavesAmpl{0.8}
    &
    \wavesPhase{0.6}
    \end{tabular}
    \vspace{20pt}
    
    \centering
    360 fokonként ($2 \pi$) periodikus; $f(x) = f(x + n \cdot 2 \pi)$, $n$ egész szám
    \end{center}
\end{frame}


\def\ft{Egy SAR felvétel fázisértékei}

\begin{frame}{\ft}
    \begin{minic}{0.8}
        \fig{sar_phase.png}
        \centering
        
        Egy SLC fázisképe.
    \end{minic}
\end{frame}


\begin{frame}{\ft}
    \begin{figp}{\fig{sar_phase.png}}{Egy SLC fázisképe.}{0.6}{0.35}
        \begin{itemize}
            \item $\phi = \frac{4\pi}{\lambda}R$; $2\pi$-ként periodikus!
            \item $R = 700-1000 $ km \hspace{10pt} $\lambda \approx 5$ cm (C-sáv)
            \item Több millió ciklus!
            \item Nem tartalmaz hasznos információt.
        \end{itemize}
    \end{figp}
\end{frame}

\def\ft{Topográfiai fázis}

\sepframe{Interferogram fázisa}

\begin{frame}{\ft}
    \begin{figp}{\inc{insar_phase.png}}{}{0.675}{0.3}
        Interferogram = két SAR felvétel fázisértékeinek különbsége:
        \begin{align*}
            \phi\{k,l\} = & A_1\{k,l\} e^{i\psi_1\{k,l\}} \times \\
                          & A_2\{k,l\} e^{-i\psi_2\{k,l\}}
        \end{align*}
    \end{figp}
\end{frame}


\begin{frame}{\ft}
    \begin{minic}{0.6}
        \inc{ifg_flattening.png}
        \centering
        
        Az interferogram \quo{kilapítása} (IFG flattening)
    \end{minic}
\end{frame}


\begin{frame}{\ft}
    \begin{figp}{\inc{burgmann_geom.png}}{Két SAR felvétel geometriája \cite{BurgmannInSAR}}{0.4}{0.5}
        \begin{itemize}
            \item $\phi = \frac{4\pi}{\lambda}\rho_1\left[\left(1 - \frac{2 \vec{B}\hat{l}_1}{\rho_1} + \frac{B^2}{\rho_1^2}\right)^2\right]$
            \item ha $B \ll \rho_{1,2}$ akkor $$\phi \approx - \frac{4\pi}{\lambda} (\vec{B}\hat{l}_1) = - \frac{4\pi}{\lambda} B \sin(\theta - \alpha)$$
            \item a magasság pedig: $$h = H - \rho_1 \cos\theta$$
            \item magassági \quo{bizonytalanság} (height ambiguity): $$\subt{h}{a} = 2\pi \pard{h}{\phi} = \frac{\lambda\rho_1\sin\theta}{2B\cos(\theta - \alpha)}$$
        \end{itemize}
    \end{figp}
\end{frame}


\begin{frame}{\ft}
    \only<1>{
    \begin{minic}{0.75}
        \fig{hooper_topo_1.png}
    \end{minic}
    }
    \only<2>{
    \begin{minic}{0.75}
        \fig{hooper_topo_2.png}
    \end{minic}
    }
    \centering
    
    Topográfiai fázisértékek két különböző merőleges bázisvonal esetén.
\end{frame}

\def\ft{Deformációs fázis}

\begin{frame}{\ft}
    \begin{figp}{\inc{burgmann_geom.png}}{Két SAR felvétel geometriája \cite{BurgmannInSAR}}{0.4}{0.5}
        \begin{itemize}
            \item $\phi \approx \frac{4\pi}{\lambda} (- \vec{B}\hat{l}_1 + \vec{D}\hat{l}_1))$
            \item $\delta\rho = \vec{D}\hat{l}_1$
            \item topográfiai fázis eltávolítása DEM segítségével; maradék fázis ideális esetben a deformációs fázis $$\phi \approx \frac{4\pi}{\lambda}\delta\rho$$
            \item CSAK műholdirányú elmozdulások detektálhatóak
        \end{itemize}
    \end{figp}
\end{frame}


\def\ft{Egyéb fázistagok}

\begin{frame}{\ft}
    \begin{minic}{0.75}
        \fig{kilauea.png}
    \end{minic}
\end{frame}

\begin{frame}{\ft}
    $$ \phi = \subt{\phi}{DEM} + \subt{\phi}{def} + \subt{\phi}{atmo} + \subt{\phi}{iono} + \subt{\phi}{pálya} + \subt{\phi}{zaj} $$
    \begin{table}[H]
        \centering
        \begin{tabular}{l l l} \toprule
            Fázistag & térbeli jellemző & időbeli jellemző\\ \midrule
            $\phi_{\text{def}}$ & hosszú hullámhosszú & alacsony frekvenciájú\\
            $\phi_{\text{DEM}}$ & hosszú hullámhosszú & bázisvonallal korrelált \\
            $\phi_{\text{atmo}}$ & hosszú hullámhosszú & magas frekvenciájú\\
            $\phi_{\text{pálya}}$ & hosszú hullámhosszú & magas frekvenciájú \\
            $\phi_{\text{zaj}}$ & rövid hullámhosszú & magas frekvenciájú \\ \bottomrule
        \end{tabular}
    \end{table}
    \centering
    
    \cite{Hooper2012}
\end{frame}

\def\ft{Interferogram fázis koherenciája}

\sepframe{Interferogram koherenciája és szűrése}

\begin{frame}{\ft}
    \begin{minic}{0.75}
        \inc{insar/theory/kilauea.png}
    \end{minic}
\end{frame}


\begin{frame}{\ft}
    % agram2015noise
    \begin{minipage}[c]{0.5\textwidth}
        \inc{insar/theory/insar_coherence.png}
    \end{minipage}
    \hspace{10pt}
    \begin{minipage}[c]{0.4\textwidth}
    Koherencia definíciója:
    $$ \gamma = \frac{|\langle s_1 s_2^* \rangle|}{\sqrt{\langle s_1 s_1^*
        \rangle \langle s_2 s_2^* \rangle}}, $$
    $s_1$, $s_2$ a két felvétel komplex pixelértékei. Átlagolás csúszóablakkal (boxcar) vagy egyéb módszerrel.
    \end{minipage}
\end{frame}


\def\ft{A fázis szűrése}

\begin{frame}{\ft}
    % agram2015noise
    \begin{minipage}[c]{0.5\textwidth}
        \inc{insar/theory/insar_coherence.png}
    \end{minipage}
    \hspace{10pt}
    \begin{minipage}[c]{0.4\textwidth}
    Goldstein-szűrés:
    \[
        H(u,v) = Z(u,v) \abs{Z(u,v)}^{\alpha}
    \]
    \begin{itemize}
        \item $H(u,v)$: szűrt értékek
        \item $Z(u,v)$: eredeti értékek
        \item $\alpha$: szűrés erőssége; $\alpha = 0$ nincs szűrés
    \end{itemize}
    \end{minipage}
\end{frame}

\def\ft{Fáziskicsomagolás}

\sepframe{\ft}

\begin{frame}{\ft}
    \begin{minipage}[c]{0.475\textwidth}
        \inc{phase_unwrapping.jpg}
    \end{minipage}
    \begin{minipage}[c]{0.475\textwidth}
        \inc{unwrap1.png}
    \end{minipage}
    \vspace{10pt}
    \begin{minipage}[c]{0.475\textwidth}
    \begin{itemize}
        \item fázis $2\pi$-ként periodukus
        \item $\pi$-nél nagyobb \quo{ugrásokat} korrigálni kell
        \item ez a lépés a fáziskicsomagolás
        \item nem triviális és nem lineáris probléma
        \item $k_n$ egész számok meghatározása
    \end{itemize}
    \end{minipage}
    \begin{minipage}[c]{0.475\textwidth}
        \inc{unwrap2.png}
    \end{minipage}
\end{frame}


\begin{frame}{\ft}
    \begin{figp}{\fig{wrapped_and_unwrapped_ifg.jpg}}{}{0.6}{0.35}
        \begin{itemize}
            \item interferogram 2D-s
            \item bonyolultabb probléma; több lehetőség ellenőrzésre
            \item viszonyítási pont kiválasztása
            \item deformáció meghatározása relatív egy kijelölt ponthoz képest: $\delta\rho = -\frac{\lambda}{4\pi}\subt{\phi}{def}$
        \end{itemize}
    \end{figp}
\end{frame}

\def\ft{Földrengés modellezés InSAR adatokkal \cite{baer2008evolution}}

\sepframe{\ft}

\begin{frame}{\ft}
    \begin{minipage}[c]{0.52\textwidth}
        \inc{insar/theory/eq_area.png}
    \end{minipage}
    \hspace{5pt}
    \begin{minipage}[c]{0.45\textwidth}
        \inc{insar/theory/eq_insar2.png}
        \vspace{10pt}
        
        Envisat interferogramok.
        \vspace{10pt}
        
        Tanulmányban vizsgált terület: Kenya Rift Zóna.
    \end{minipage}
\end{frame}


\begin{frame}{\ft}
    \begin{minic}{0.8}
        \inc{insar/theory/eq_insar2.png}
    \end{minic}
    \centering
    
    Envisat interferogramok.
\end{frame}

\begin{frame}{\ft}
    \begin{minipage}[c]{0.4\textwidth}
        \inc{insar/theory/eq_insar_model.png}
        
    \end{minipage}
    \hspace{15pt}
    \begin{minipage}[c]{0.4\textwidth}
        \inc{insar/theory/eq_insar_model_results.png}
        
        Dike betüremkedés modellek eredményei.
    \end{minipage}
\end{frame}

\def\ft{Deformációs idősor előállítása}

\sepframe{\ft}

\begin{frame}{\ft}
    \begin{figp}{\inc{insar/theory/SBAS_vs_PSI.jpg}}{}{0.7}{0.25}
        \begin{itemize}
            \item minden kör egy SAR felvétel; minden vonal egy interferogram
            \item PS: Persistent Scatterer
            \item SBAS: Small Baseline Subset
        \end{itemize}
    \end{figp}
    \vspace{25pt}
    \centering
    
    Interferogramok térbeli 2D kicsomagolás, 1D időbeli kicsomagolás.
\end{frame}


\begin{frame}{\ft}
    \begin{minic}{0.8}
        \inc{insar/theory/scatterers.png}
    \end{minic}
    \centering
    \begin{itemize}
        \item a: egyedi szórópont (single scatterer)
        \item b: domináns szórópont (domiant scatterer)
        \item c: elosztott koherens szórópontok (coherent distributed scatterers)
        \item d: elosztott inkoherens szórópontok (incoherent distributed scatterers)
    \end{itemize}
    PS: egyedi- és domináns szórópontok $\leftrightarrow$ SBAS: elosztott koherens szórópontok
\end{frame}

% 

\begin{frame}{\ft}
    \begin{minic}{1.0}
        \inc{insar/theory/ps_sbas_pixels.png}
    \end{minic}
    \centering
    
    PS és SBAS módszer által kiválasztott pixelek. Eyjafjallajökull-vulkán, Izland. \cite{hooper2008multi}
\end{frame}

\begin{frame}{\ft}
    \begin{minic}{0.6}
        \inc{insar/theory/iceland_ts.png}
    \end{minic}
    \centering
    
    PS és SBAS módszer együttes alkalmazása során kapott deformációs idősor. Eyjafjallajökull-vulkán, Izland. \cite{hooper2008multi}
\end{frame}


\def\ft{Felszálló és leszálló felvételek kombinálása}

\sepframe{\ft}

\begin{frame}{\ft}
    \begin{minic}{0.70}
        \inc{insar/theory/ascending_descending_orbits.png}
    \end{minic}
    \centering
    
    Felszálló (ascending) és leszálló (descending) műholdpálya.
\end{frame}

\begin{frame}{\ft}
    \begin{minipage}[c]{0.6\textwidth}
        \inc{insar/theory/daisy_schematic.png}
    \end{minipage}
    \begin{minipage}[c]{0.35\textwidth}
        \begin{itemize}
            \item felszálló és leszálló felvétel $\rar$ két különböző LOS vektor
            \item a két vektor kb. a kelet-nyugat és vertikális síkban
            \item felszálló és leszálló deformációs sebességek kombinálása $\rar$ kelet-nyugati és vertikális sebességkomponensek
        \end{itemize}
    \end{minipage}
\end{frame}

\begin{frame}{\ft}
    \begin{minipage}[c]{0.4\textwidth}
        \inc{insar/reflectors/networks/dszekcso/network.png}
    \end{minipage}
    \hspace{0.025\textwidth}
    \begin{minipage}[c]{0.55\textwidth}
        \inc{insar/reflectors/networks/dszekcso/refl_1.jpg}
        \centering
        
        Dunaszekcsői reflektorhálózat
    \end{minipage}
\end{frame}

\begin{frame}{\ft}
    \begin{minipage}[c]{0.55\textwidth}
        \inc{insar/reflectors/networks/dszekcso/IB2-IB1_kalman.png}
    \end{minipage}    
    \begin{minipage}[c]{0.35\textwidth}
        \inc{insar/reflectors/networks/dszekcso/IB2-IB1_los.png}
    \end{minipage}    
\end{frame}

\begin{frame}{\ft}
\end{frame}


%\begin{frame}{\ft}
    %\begin{minic}{0.8}
        %\inc{insar/theory/dp_vectors.png}
    %\end{minic}
    %\centering
    
    %Envisat interferogramok alapján számított kele-nyugati és vertikális sebességkomponensek.
%\end{frame}


\begin{frame}{Összefoglalás}
    \begin{itemize}
        \item InSAR, nagy felszíni felbontással felszíni elmozdulások térképezése
        \item jó pár feldolgozási lépés amiről nem beszéltem
        \item elektromágneses hullámterjedés, jelfeldolgozás (Fourier-transzformáció, digitális szűrés), adatfeldolgozási módszerek
    \end{itemize}
\end{frame}

\begin{frame}
    \begin{center}
        \Huge \color{blue!55!black}
        Köszönöm a figyelmet!
    \end{center}
\end{frame}

\begin{frame}[allowframebreaks]{Hivatkozáslista}
    \bibliography{insar}
\end{frame}


\end{document}
