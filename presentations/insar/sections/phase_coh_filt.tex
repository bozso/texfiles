\def\ft{Interferogram fázis koherenciája}

\sepframe{Interferogram koherenciája és szűrése}

\begin{frame}{\ft}
    \begin{minic}{0.75}
        \inc{insar/theory/kilauea.png}
    \end{minic}
\end{frame}


\begin{frame}{\ft}
    % agram2015noise
    \begin{minipage}[c]{0.5\textwidth}
        \inc{insar/theory/insar_coherence.png}
    \end{minipage}
    \hspace{10pt}
    \begin{minipage}[c]{0.4\textwidth}
    Koherencia definíciója:
    $$ \gamma = \frac{|\langle s_1 s_2^* \rangle|}{\sqrt{\langle s_1 s_1^*
        \rangle \langle s_2 s_2^* \rangle}}, $$
    $s_1$, $s_2$ a két felvétel komplex pixelértékei. Átlagolás csúszóablakkal (boxcar) vagy egyéb módszerrel.
    \end{minipage}
\end{frame}


\def\ft{A fázis szűrése}

\begin{frame}{\ft}
    % agram2015noise
    \begin{minipage}[c]{0.5\textwidth}
        \inc{insar/theory/insar_coherence.png}
    \end{minipage}
    \hspace{10pt}
    \begin{minipage}[c]{0.4\textwidth}
    Goldstein-szűrés:
    \[
        H(u,v) = Z(u,v) \abs{Z(u,v)}^{\alpha}
    \]
    \begin{itemize}
        \item $H(u,v)$: szűrt értékek
        \item $Z(u,v)$: eredeti értékek
        \item $\alpha$: szűrés erőssége; $\alpha = 0$ nincs szűrés
    \end{itemize}
    \end{minipage}
\end{frame}
