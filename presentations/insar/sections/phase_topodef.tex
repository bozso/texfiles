\def\ft{Topográfiai fázis}

\sepframe{Interferogram fázisa}

\begin{frame}{\ft}
    \begin{figp}{\inc{insar_phase.png}}{}{0.675}{0.3}
        Interferogram = két SAR felvétel fázisértékeinek különbsége:
        \begin{align*}
            \phi\{k,l\} = & A_1\{k,l\} e^{i\psi_1\{k,l\}} \times \\
                          & A_2\{k,l\} e^{-i\psi_2\{k,l\}}
        \end{align*}
    \end{figp}
\end{frame}


\begin{frame}{\ft}
    \begin{minic}{0.6}
        \inc{ifg_flattening.png}
        \centering
        
        Az interferogram \quo{kilapítása} (IFG flattening)
    \end{minic}
\end{frame}


\begin{frame}{\ft}
    \begin{figp}{\inc{burgmann_geom.png}}{Két SAR felvétel geometriája \cite{BurgmannInSAR}}{0.4}{0.5}
        \begin{itemize}
            \item $\phi = \frac{4\pi}{\lambda}\rho_1\left[\left(1 - \frac{2 \vec{B}\hat{l}_1}{\rho_1} + \frac{B^2}{\rho_1^2}\right)^2\right]$
            \item ha $B \ll \rho_{1,2}$ akkor $$\phi \approx - \frac{4\pi}{\lambda} (\vec{B}\hat{l}_1) = - \frac{4\pi}{\lambda} B \sin(\theta - \alpha)$$
            \item a magasság pedig: $$h = H - \rho_1 \cos\theta$$
            \item magassági \quo{bizonytalanság} (height ambiguity): $$\subt{h}{a} = 2\pi \pard{h}{\phi} = \frac{\lambda\rho_1\sin\theta}{2B\cos(\theta - \alpha)}$$
        \end{itemize}
    \end{figp}
\end{frame}


\begin{frame}{\ft}
    \only<1>{
    \begin{minic}{0.75}
        \fig{hooper_topo_1.png}
    \end{minic}
    }
    \only<2>{
    \begin{minic}{0.75}
        \fig{hooper_topo_2.png}
    \end{minic}
    }
    \centering
    
    Topográfiai fázisértékek két különböző merőleges bázisvonal esetén.
\end{frame}

\def\ft{Deformációs fázis}

\begin{frame}{\ft}
    \begin{figp}{\inc{burgmann_geom.png}}{Két SAR felvétel geometriája \cite{BurgmannInSAR}}{0.4}{0.5}
        \begin{itemize}
            \item $\phi \approx \frac{4\pi}{\lambda} (- \vec{B}\hat{l}_1 + \vec{D}\hat{l}_1))$
            \item $\delta\rho = \vec{D}\hat{l}_1$
            \item topográfiai fázis eltávolítása DEM segítségével; maradék fázis ideális esetben a deformációs fázis $$\phi \approx \frac{4\pi}{\lambda}\delta\rho$$
            \item CSAK műholdirányú elmozdulások detektálhatóak
        \end{itemize}
    \end{figp}
\end{frame}


\def\ft{Egyéb fázistagok}

\begin{frame}{\ft}
    \begin{minic}{0.75}
        \fig{kilauea.png}
    \end{minic}
\end{frame}

\begin{frame}{\ft}
    $$ \phi = \subt{\phi}{DEM} + \subt{\phi}{def} + \subt{\phi}{atmo} + \subt{\phi}{iono} + \subt{\phi}{pálya} + \subt{\phi}{zaj} $$
    \begin{table}[H]
        \centering
        \begin{tabular}{l l l} \toprule
            Fázistag & térbeli jellemző & időbeli jellemző\\ \midrule
            $\phi_{\text{def}}$ & hosszú hullámhosszú & alacsony frekvenciájú\\
            $\phi_{\text{DEM}}$ & hosszú hullámhosszú & bázisvonallal korrelált \\
            $\phi_{\text{atmo}}$ & hosszú hullámhosszú & magas frekvenciájú\\
            $\phi_{\text{pálya}}$ & hosszú hullámhosszú & magas frekvenciájú \\
            $\phi_{\text{zaj}}$ & rövid hullámhosszú & magas frekvenciájú \\ \bottomrule
        \end{tabular}
    \end{table}
    \centering
    
    \cite{Hooper2012}
\end{frame}
