\def\ft{Ismétlés - Szintetikus Apertúrájú Radar}

\begin{frame}{\ft}
    \begin{itemize}
        \item kiváló térbeli felbontás, egy csatornán (mikrohullámú tartomány)
        \begin{itemize}
            \item L-sáv: 1-2 GHz; 30-15 cm (ALOS-PALSAR (JAXA))
            \item S-sáv: 2-4 GHz; 15-7.5 cm
            \item C-sáv: 4-8 GHz; 7.5-3.8 cm (Sentinel-1 (ESA), ERS (ESA), Envisat (ESA))
            \item X-sáv: 8-12 GHz; 3.82.5 cm (TerraSAR-X (DLR), TanDEM-X(DLR))
        \end{itemize}
        \item koherens radar: amplitúdó ÉS fázis regisztrálása
        \item alkalmazás amplitúdóra: Freidl Zoli előadása
    \end{itemize}
\end{frame}


\begin{frame}{\ft}
    % \cite{RamonHanssen}
    \begin{figp}{\inc{insar/theory/ramon_SAR.png}}{Radar felvétel geometriája \cite{RamonHanssen}}{0.5}{0.4}
        RAR (Real Aperture Radar) felbontása: azimut / slant range $\approx 5$ km, ground range $\approx 14$ km:
        $$ \subt{W}{a} = \frac{\lambda}{\subt{L}{a}} R $$
        $$ \subt{\Delta}{r} = \frac{c\tau}{2} $$
    \end{figp}
\end{frame}

\def\ft{Range felbontás növelése}

\begin{frame}{\ft}
    \begin{figp}{\inc{insar/theory/chirp_demod.png}}{}{0.4}{0.4}
        Chirp jel kibocsájtása, visszavert jel feldolgozása illesztett szűréssel:
        \begin{align*}
            f(t) &= a t & \text{lineárisan növekvő frekvencia} \\
            \subt{B}{R} &= a \tau & \text{sávszélesség} \\
            \subt{\Delta}{r} &= \frac{c}{2 \subt{B}{R}} \approx 9.6 m & \text{slant range felbontás}
        \end{align*}
        $\approx 25$ m ground range felbontás
    \end{figp}
\end{frame}

\def\ft{Azimut felbontás növelése}

\begin{frame}{\ft}
    \begin{center}
    \begin{minipage}[b]{0.4\textwidth}
        \inc{insar/theory/sar_synthetic_length.png}
        \centering
        
        Egy földfelszíni szórópont leképződése a SAR felvétel készítése során.
    \end{minipage}
    \hspace{10pt}
    \begin{minipage}[b]{0.5\textwidth}
        \inc{insar/theory/chirp_azimuth.png}
        \centering
        
        A földfelszíni szórópont által visszavert chirp jelek.
    \end{minipage}
    \end{center}
\end{frame}


\def\ft{Nyers SAR felvétel fókuszálása}

\begin{frame}{\ft}
    \begin{minic}{0.70}
        \inc{insar/theory/rng_azi_compression.png}
        
        \centering
        Felbontás fókuszálás után: ground range: $\approx 25$ m, azimut: $\approx 10$ m (ERS-1,2)
    \end{minic}
\end{frame}


\def\ft{Ismétlés - Szintetikus Apertúrájú Radar}

\begin{frame}{\ft}
    \begin{minic}{0.80}
        \inc{insar/theory/mw_bands.png}
        
        \centering
        Különböző radarsávban készült felvételek egy területről.
    \end{minic}
\end{frame}


\begin{frame}{\ft}
    % insartut
    \begin{figp}{\inc{insar/theory/sar_speckle.png}}{}{0.45}{0.45}
        \begin{itemize}
            \item amplitúdó ÉS fázis érték pixelenként; komplex számként reprezentálható
            \item egyetlen SAR pixel értéke: $\subt{C}{SAR} = A e^{i\phi}$
            \item $\subt{C}{SAR} = \sum_j c_j$
        \end{itemize}
    \end{figp}
\end{frame}


\begin{frame}{\ft}
    \begin{minic}{0.64}
        %RamonHanssen
        \inc{insar/theory/ramon_shortening.png}
        \centering
        
        Topográfia által okozott geometriai torzulásai a felbontási cellának.
    \end{minic}
\end{frame}


\begin{frame}{\ft}
    \begin{minic}{0.85}
        \inc{insar/theory/mt_merapi.png}
        \centering
        
        Különböző felvételi geometriából készült felvételek a Merapi-hegyről (Indonézia)
    \end{minic}
\end{frame}
