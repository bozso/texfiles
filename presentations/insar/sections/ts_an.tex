\def\ft{Deformációs idősor előállítása}

\sepframe{\ft}

\begin{frame}{\ft}
    \begin{figp}{\inc{SBAS_vs_PSI.jpg}}{}{0.7}{0.25}
        \begin{itemize}
            \item minden kör egy SAR felvétel; minden vonal egy interferogram
            \item PS: Persistent Scatterer
            \item SBAS: Small Baseline Subset
        \end{itemize}
    \end{figp}
    \vspace{25pt}
    \centering
    
    Interferogramok térbeli 2D kicsomagolás, 1D időbeli kicsomagolás.
\end{frame}


\begin{frame}{\ft}
    \begin{minic}{0.8}
        \inc{scatterers.png}
    \end{minic}
    \centering
    \begin{itemize}
        \item a: egyedi szórópont (single scatterer)
        \item b: domináns szórópont (domiant scatterer)
        \item c: elosztott koherens szórópontok (coherent distributed scatterers)
        \item d: elosztott inkoherens szórópontok (incoherent distributed scatterers)
    \end{itemize}
    PS: egyedi- és domináns szórópontok $\leftrightarrow$ SBAS: elosztott koherens szórópontok
\end{frame}

% 

\begin{frame}{\ft}
    \begin{minic}{1.0}
        \inc{ps_sbas_pixels.png}
    \end{minic}
    \centering
    
    PS és SBAS módszer által kiválasztott pixelek. Eyjafjallajökull-vulkán, Izland. \cite{hooper2008multi}
\end{frame}

\begin{frame}{\ft}
    \begin{minic}{0.6}
        \inc{iceland_ts.png}
    \end{minic}
    \centering
    
    PS és SBAS módszer együttes alkalmazása során kapott deformációs idősor. Eyjafjallajökull-vulkán, Izland. \cite{hooper2008multi}
\end{frame}


\def\ft{Felszálló és leszálló felvételek kombinálása}

\sepframe{\ft}

\begin{frame}{\ft}
    \begin{minic}{0.75}
        \inc{ascending_descending_orbits.png}
    \end{minic}
    \centering
    
    Felszálló (ascending) és leszálló (descending) műholdpálya.
\end{frame}

\begin{frame}{\ft}
    \begin{minipage}[c]{0.6\textwidth}
        \inc{daisy_schematic.png}
    \end{minipage}
    \begin{minipage}[c]{0.35\textwidth}
        \begin{itemize}
            \item felszálló és leszálló felvétel $\rar$ két különböző LOS vektor
            \item a két vektor kb. a kelet-nyugat és vertikális síkban
            \item felszálló és leszálló deformációs sebességek kombinálása $\rar$ kelet-nyugati és vertikális sebességkomponensek
        \end{itemize}
    \end{minipage}
\end{frame}


\begin{frame}{\ft}
    \begin{minic}{0.8}
        \inc{dp_vectors.png}
    \end{minic}
    \centering
    
    Envisat interferogramok alapján számított kele-nyugati és vertikális sebességkomponensek.
\end{frame}
