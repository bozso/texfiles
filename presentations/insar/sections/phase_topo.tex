\def\ft{Topográfiai fázis}
\subsection{\ft}

\begin{frame}{\ft}
    \begin{figp}{\inc{insar_phase.png}}{}{0.675}{0.3}
        Interferogram = két SAR felvétel fázisértékeinek különbsége:
        %\begin{align*}
            %\phi\{k,l\} = & A_1\{k,l\} e^{i\psi_1\{k,l\}} \cdot \\
                          %& A_2\{k,l\} e^{-i\psi_2\{k,l\}}
        %\end{align*}
    \end{figp}
\end{frame}


\begin{frame}{\ft}
    \begin{minic}{0.6}
        \inc{ifg_flattening.png}
        \centering
        
        Az interferogram \quo{kilapítása} (IFG flattening)
    \end{minic}
\end{frame}


\begin{frame}{\ft}
    \begin{figp}{\inc{burgmann_geom.png}}{Két SAR felvétel geometriája}{0.4}{0.5}
        \begin{itemize}
            \item $\phi = \frac{4\pi}{\lambda}\rho_1\left[\left(1 - \frac{2 \vec{B}\hat{l}_1}{\rho_1} + \frac{B^2}{\rho_1^2}\right)^2\right]$
            \item ha $B \ll \rho_{1,2}$ akkor $$\phi \approx - \frac{4\pi}{\lambda} (\vec{B}\hat{l}_1) = - \frac{4\pi}{\lambda} B \sin(\theta - \alpha)$$
            \item a magasság pedig: $$h = H - \rho_1 \cos\theta$$
            \item magassági \quo{bizonytalanság} (height ambiguity): $$\subt{h}{a} = 2\pi \pard{h}{\phi} = \frac{\lambda\rho_1\sin\theta}{2B\cos(\theta - \alpha)}$$
        \end{itemize}
    \end{figp}
\end{frame}


\begin{frame}{\ft}
    \begin{center}
    \begin{minipage}[c]{0.45\textwidth}
        \fig{hooper_topo_1.png}
    \end{minipage}
    \hspace{10pt}
    \begin{minipage}[c]{0.45\textwidth}
        \fig{hooper_topo_2.png}
    \end{minipage}
    \centering
    
    Topográfiai fázisértékek két különböző merőleges bázisvonal esetén.
    \end{center}
\end{frame}
