\def\ft{Földrengés modellezés InSAR adatokkal \cite{baer2008evolution}}

\sepframe{\ft}

\begin{frame}{\ft}
    \begin{minipage}[c]{0.52\textwidth}
        \inc{insar/theory/eq_area.png}
    \end{minipage}
    \hspace{5pt}
    \begin{minipage}[c]{0.45\textwidth}
        \inc{insar/theory/eq_insar2.png}
        \vspace{10pt}
        
        Envisat interferogramok.
        \vspace{10pt}
        
        Tanulmányban vizsgált terület: Kenya Rift Zóna.
    \end{minipage}
\end{frame}


\begin{frame}{\ft}
    \begin{minic}{0.8}
        \inc{insar/theory/eq_insar2.png}
    \end{minic}
    \centering
    
    Envisat interferogramok.
\end{frame}

\begin{frame}{\ft}
    \begin{minipage}[c]{0.4\textwidth}
        \inc{insar/theory/eq_insar_model.png}
        
    \end{minipage}
    \hspace{15pt}
    \begin{minipage}[c]{0.4\textwidth}
        \inc{insar/theory/eq_insar_model_results.png}
        
        Dike betüremkedés modellek eredményei.
    \end{minipage}
\end{frame}
