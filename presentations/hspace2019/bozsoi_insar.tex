\documentclass{beamer}
\usepackage[english]{babel}
\usepackage[utf8]{inputenc}
\usepackage[T1]{fontenc}
\usepackage{graphicx, tikz}
\usepackage[framemethod=TikZ]{mdframed}

\mode<presentation>
{
    \usetheme{Boadilla}
    \usecolortheme{default}
    \usefonttheme{default}
    \setbeamertemplate{navigation symbols}{}
    \setbeamertemplate{caption}[numbered]
} 


\title[H-SPACE, Budapest 2019]{Exploitation of Sentinel-1 SAR data for studying geodynamic, tropospheric and ionospheric processes.}
\author[Bozsó et al.]{István Bozsó, Eszter Szűcs, László Bányai, Viktor Wesztergom}
\institute[MTA CSFK GGI]{MTA CSFK Geodetic and Geophysical Institute}
\date{2019.02.28.}

\def\rar{\rightarrow}
\def\lar{\leftarrow}


\newcommand\inc[1] {
    \includegraphics[width=\textwidth]{#1}
}

\newcommand{\ffig}[1]{
    \begin{mdframed}[linecolor=blue!65!white, linewidth=1.75pt, roundcorner=0.75pt,
                     innerrightmargin=0pt, innerleftmargin=0pt,
                     innertopmargin=0pt, innerbottommargin=0pt,
                     backgroundcolor=white, frametitle={}, align=center]
        \includegraphics[width=1.0\textwidth]{#1}
    \end{mdframed}
}


\newcommand\sepframe[1] {
    \begin{frame}
        \begin{center}
            \Huge \color{blue!55!black}
            #1
        \end{center}
    \end{frame}
}


\graphicspath{{../../images/}}


\begin{document}

\begin{frame}
    \titlepage
\end{frame}

\begin{frame}{Introduction}
    InSAR technology: Synthetic Aperture Radar (SAR) based interferometry:
    \begin{itemize}
        \item SAR images: ground-based, airplane, UAV, satellite
        \item high resolution; $20 \mathrm{m} \times 5 \mathrm{m}$ sized pixels - Low Earth Orbit C-band (5.405 MHz) SAR satellites
        \item each pixel: amplitude and phase represented as a complex number
        \item single scene phases are random
        \item formation of an interferogram: phase differences between two SAR scenes
    \end{itemize}
\end{frame}

\begin{frame}{Interferogram}
    \begin{center}
        \includegraphics[width=0.8\textwidth]{insar_phase.png}
    \end{center}
\end{frame}

\begin{frame}{Phase of the interferogram}
    \[
        \subt{\Phi}{IFG} = \subt{\Phi}{defo.} + \subt{\Phi}{atmo.} + \subt{\Phi}{topo.} + \subt{\Phi}{orbit} + \subt{\Phi}{noise}
    \]
    \begin{itemize}
        \item $\subt{\Phi}{IFG}$: phase of the interferogram 
        \item $\subt{\Phi}{defo.}$ : phase caused by surface deformation, that occurred between the two SAR acquisitions, in the Line-of-Sight (LOS) direction
        \item $\subt{\Phi}{atmo.}$: phase caused by the change in atmospheric microwave propagation speed
        \item $\subt{\Phi}{topo.}$: phase due to topography
        \item $\subt{\Phi}{orbit}$: phase from errors of the orbital state vector
        \item $\subt{\Phi}{noise}$: noise from residual phase terms, that cannot be modeled
    \end{itemize}    
\end{frame}

\begin{frame}{Processing the interferometric phase}
    \[
        \subt{\Phi}{IFG} = \subt{\Phi}{defo.} + \subt{\Phi}{atmo.} + \subt{\Phi}{topo.} + \subt{\Phi}{orbit} + \subt{\Phi}{noise}
    \]
    Separation and estimation of the different phase terms:
    \begin{itemize}
        \item $\subt{\Phi}{topo.}$: using DEM, residuals correlated with baseline
        \item $\subt{\Phi}{atmo.}$: atmopsheric models, temporal filtering
        \item $\subt{\Phi}{oribt}$: more precise orbit data, deramping
        \item spatial filtering Goldstein-filter %\ref{}
    \end{itemize}
\end{frame}


\begin{frame}{Deformation}
    If estimation of $\subt{\Phi}{defo.}$ is done $\rar$ phase unwrapping $\rar$ LOS deformation.

    \begin{center}
        \includegraphics[width=0.55\textwidth]{wrapped_and_unwrapped_ifg.jpg}
    \end{center}
    \[
        \subt{d}{defo.} = \frac{\lambda}{4\pi} \subt{\Phi}{defo.,unwrapped}
    \]
    Multiple interferograms $\rar$ time-series analysis $\rar$ LOS deformation time-series.
\end{frame}


\def\ft{Mapping surface displacement based on natural scatterers}
\sepframe{Examples of InSAR applications: \ft}


\begin{frame}{\ft}
    \begin{itemize}
        \item $\approx$ 3.5 years of Sentinel-1 A descending orbit data (105 scenes)
        \item covering the Praid salt extrusion in Carpathian Bend area
        \item salt deformation governed by weather phenomena
        \item TopoTransyvania project: investigation of the Carpathian Bend and subduction zone
        \item analyse geodynamic processes (seismic activity, post volcanic activity, salt tectonics)
    \end{itemize}
\end{frame}


\begin{frame}{\ft}
    \begin{center}
        \includegraphics[width=0.6\textwidth]{parajd.png}
    \end{center}
    First results show deformation on the southern flanks of the salt diapir. Subsidence in the $3-4$ cm/yr  range. Lack of scatterers on the top of the salt diapir $\rar$ artificial reflectors.
\end{frame}


\begin{frame}{\ft}
    \begin{center}
        \includegraphics[width=0.6\textwidth]{parajd_ts.png}
    \end{center}
\end{frame}


\def\ft{Monitoring the displacement time-series of benchmark reflector networks}

\sepframe{Examples of InSAR applications: \ft}

\begin{frame}{\ft}
    \begin{itemize}
        \item $\approx$ 1 year of Sentinel-1 A and B ascending and descending orbit data
        \item geodetic/geodynamic integrated benchmarks (IBs)
        \item IB network, settlement (Dunaszekcső) along the loess banks of the Danube
        \item GNSS measurements 1 year apart
        \item 3D displacements from combination of InSAR and GNSS data with Kalman filtering
    \end{itemize}

    \begin{center}
        \figm{dszekcso_refl_1.jpg}{0.5}
    \end{center}
\end{frame}


\begin{frame}{\ft}
    \begin{center}
        \includegraphics[width=0.9\textwidth]{dszekcso_ts.png}
    \end{center}
\end{frame}

\end{document}
