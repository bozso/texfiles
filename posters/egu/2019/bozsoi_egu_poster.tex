\documentclass[a0, 18pt, landscape]{a0poster}
\usepackage[english]{babel}
\usepackage[utf8]{inputenc}
\usepackage[T1]{fontenc}


\usepackage{graphicx, tikz, helvet, pagecolor, multicol, siunitx, epstopdf}

%\usepackage[colorlinks]{hyperref}
\usepackage[framemethod=tikz]{mdframed}
\usepackage[left=25pt, right=25pt, top=25pt, bottom=25pt]{geometry}
\usepackage[font=small,labelfont=bf]{caption}


\graphicspath{{/home/istvan/Dokumentumok/texfiles/images/}}

\pagecolor{blue!5!white}


\definecolor{bcol}{RGB}{0,100,125}
\definecolor{tcol}{RGB}{0,200,225}
\definecolor{fcol}{RGB}{80,45,45}
\definecolor{acol}{RGB}{255,150,0}
\definecolor{abcol}{RGB}{0,250,250}
\definecolor{ftback}{RGB}{0,200,250}


\def\rar{\rightarrow}
\def\lar{\leftarrow}


\newcommand\inc[1] {
    \includegraphics[width=\textwidth]{#1}
}

\newenvironment{block}[2]
{
    \begin{mdframed}[linecolor=#2, linewidth=6.0pt, roundcorner=12.5pt,
                     innerrightmargin=15pt, innerleftmargin=15pt,
                     innertopmargin=15pt, innerbottommargin=15pt,
                     backgroundcolor=white, frametitle={}, align=center]
    \vspace{15pt}
    
    \begin{center}
        { \Huge \underline{\textbf{#1}} }
    \end{center}
    
    \vspace{15pt}
}
{
    \end{mdframed}
}


\def\bblock{ \begin{block} }
\def\eblock{ \end{block} }


\newenvironment{bib}[1]
{
    \begin{mdframed}[linecolor=green, linewidth=3.75pt, roundcorner=3.0pt,
                     innerrightmargin=15pt, innerleftmargin=15pt,
                     innertopmargin=5pt, innerbottommargin=5pt,
                     backgroundcolor=white, frametitle={}, align=center]
}
{
    \end{mdframed}
}


\newenvironment{ack}
{
    \begin{mdframed}[linecolor=orange, linewidth=3.75pt, roundcorner=3.0pt,
                     innerrightmargin=15pt, innerleftmargin=15pt,
                     innertopmargin=5pt, innerbottommargin=5pt,
                     backgroundcolor=white, frametitle={}, align=center]
    \vspace{15pt}
        { \normalfont \textbf{Acknowledgements} }
    \vspace{15pt}
    \newline
}
{
    \end{mdframed}
}

A: 407
X1: 58, 60, 62
X2: 195,
X4: 219, 252, 253, 254, 328



\newcommand{\finc}[1] {
    \begin{mdframed}[linecolor=fcol, linewidth=5.0pt, roundcorner=5pt,
                     innerrightmargin=0pt, innerleftmargin=0pt,
                     innertopmargin=0pt, innerbottommargin=0pt,
                     skipabove=0pt, skipbelow=0pt,
                     backgroundcolor=white, frametitle={}, align=center]
    \begin{center}
        \inc{#1}
    \end{center}
    \end{mdframed}
}


\begin{document}

%-----------------------------------------------------------------------
% TITLE
%-----------------------------------------------------------------------

\bfseries

\begin{minipage}{0.935\textwidth}
\begin{mdframed}[linecolor=black, linewidth=5pt, innerleftmargin=10pt, innerrightmargin=10pt, innertopmargin=25pt,
                 innerbottommargin=25pt, backgroundcolor=tcol, roundcorner=50pt]
    \hspace{10pt}
    \begin{minipage}[c]{0.08\textwidth}
        \includegraphics[width=\textwidth]{ggi_logo.png}
        \vspace{2.5pt}
        
        { \normalfont \bfseries EGU General Assembly 2019 -- X3.173 }
    \end{minipage}
    \begin{minipage}[c]{0.9\textwidth}
      \centering {\Huge MONITORING 3D DEFORMATION OF RADAR REFLECTORS WITH THE FUSION OF SENTINEL-1 SAR INTERFEROMETRY AND GNSS}\\[35pt] % Title
      {\Large István Bozsó, Eszter Szűcs, László Bányai, Viktor Wesztergom }\\[20pt]
     {\Large \normalfont Hungarian Academy of Sciences, Research Centre for Astronomy and Earth Sciences, Geodetic and Geophysical Institute}\\[25pt]
    {\Large \texttt{bozso.istvan@csfk.mta.hu}}
    \end{minipage}
\end{mdframed}
\end{minipage}
\hspace{15pt}
\begin{minipage}[c]{0.05\textwidth}
    \begin{center}
        \includegraphics[width=\textwidth]{graphic_egu_photo_yes.png}\\[10pt]
        \includegraphics[width=0.95\textwidth]{CreativeCommons_Attribution_License.png}
    \end{center}
\end{minipage}

\vspace{25pt}

%-----------------------------------------------------------------------
% ABSTRACT
%-----------------------------------------------------------------------

\small

\begin{mdframed}[linecolor=acol, linewidth=4pt,
    innerleftmargin=20pt, innerrightmargin=20pt,
    innerbottommargin=25pt, innertopmargin=30pt,
    backgroundcolor=abcol, roundcorner=2.5pt
    frametitleaboveskip=15pt, frametitlebelowskip=10pt,
    roundcorner=20pt, frametitle={\LARGE Abstract},
    frametitlealignment=\center]

\columnsep=30pt


\begin{multicols}{2}

Space-borne radar interferometry (InSAR) is an excellent tool for measuring surface displacements. Using the phase information stored in the SAR images it is possible to calculate the phase difference due to surface movements between two acquisitions and from the differential phase, calculate the surface displacements.
This method can be used reliably to create surface deformation time-series for mapping large areas and monitoring single reflecting targets. Some areas, however, are not suitable for InSAR analysis due to heavy vegetation or other factors that decrease phase coherence. In such cases, it is possible to install artificial reflecting objects (radar reflectors) that provide a stable phase over a long period of time.
The MTA CSFK Geodetic and Geophysical Institute in collaboration with the Department of Broadband Infocommunications and Electromagnetic Theory of the Budapest University of Technology and Economics under ESA PECS project (project ID: 4000118850/16/NL/SC) developed such artificial reflectors.
To overcome the other limitation of InSAR, namely that only deformation that is parallel to the satellite line-ofsight (LOS) can be measured, we have also carried out GNSS measurements at our reflector benchmarks.
Three reflector networks were deployed in Hungary, two along the loess banks of River Danube, where landslide activity is frequent and one near Lake Balaton in an area affected by landslides as well.
We used Sentinel-1 A and B satellite data between two GNSS campaign measurements (from July of 2016 to March of 2017) to calculate ascending and descending LOS displacements at our reflector benchmarks and through a technique called Kalman-filtering we fused data from ascending and descending satellite passes and the GNSS measurements to calculate the 3D displacements of our reflector networks, relative to a single reference reflector for each network.
We present the 3D movements we have calculated with the previously described method at the three reflector networks and comment on the potential error sources and the possible improvements to our methodology.
\end{multicols}

\end{mdframed}

\vspace{5pt}

\columnsep=30pt

\large\bfseries

\begin{multicols}{3}

\bblock{Introduction}{bcol}
    \bitz
        \item InSAR = Synthetic Aperture Radar (SAR) Interferometry
        \item ground based, air or spaceborne platform: emits and receives
              microwave signal reflected from the surface
        \item registers reflected wave amplitude and phase $\rar$ complex number, Single Look Complex image (SLC)
        \item phase difference of two SAR scenes (SLCs) $\rar$ interferogram
        \item unwrapped, filtered and corrected phase proportional to line-of-sight (LOS) surface displacement
    \eitz
    
    Interferometric phase terms:
    \[
        \subt{\Phi}{IFG} = \subt{\Phi}{defo.} + \subt{\Phi}{atmo.} + \subt{\Phi}{topo.} + \subt{\Phi}{orbit} + \subt{\Phi}{noise}
    \]
    
    \[
        \subt{\Phi}{defo.} = \frac{4\pi}{\lambda} d, \hspace{25pt} d: \text{satellite line-of-sight deformation}
    \]
    
    \bcent
    \bmini{0.475\textwidth}
    \bffig{fcol}
        \inc{DInSAR_concept.jpg}
        \fcap{Phase difference between aqcuisitions caused by satellite LOS displacement.}
    \effig
    \emini
    \bmini{0.475\textwidth}
        \inc{unwrap1.png}
        \fcap{1D unwrapping of interfeometric phase.}
    \emini
    \ecent
    \vspace{10pt}
    
    \bitz
        \item $\subt{\Phi}{IFG}$: phase of the interferogram 
        \item $\subt{\Phi}{defo.}$ : phase due to surface deformation, that occurred between the two SAR acquisitions, in the Line-of-Sight (LOS) direction
        \item $\subt{\Phi}{atmo.}$: phase due to change in atmospheric microwave propagation speed
        \item $\subt{\Phi}{topo.}$: phase due to topography
        \item $\subt{\Phi}{orbit}$: phase from errors of the orbital state vector
        \item $\subt{\Phi}{noise}$: noise adn/or unmodeled from residual phase terms
    \eitz
    
    Phase unwrapping: phase only known modulo 2$\pi$. Search for correct $n$ ($n \cdot 2\pi$ phase cycles,
    $n$ is an integer) addition factors. Can and often will fail in case of high deformation
    gradients (both in space and time).
\eblock


\bblock{Methodology}{bcol}
    \bitz
        \item Integrated Reflector benchmarks are needed, where InSAR coherence is unsatisfactory
              for reliable LOS displacement calculation.
        \item Sentinel-1 SLC scenes preprocessed (coregistration, geocoding, interferogram formation)
              with the Gamma Software \cite{werner2000gamma}.
        \item Selection of point scatterer (PS) and distributed scatterer (DS) points and unwrapping done by StaMPS Software \cite{Hooper2008}.
        \item Combination of LOS deformation from ascending (ASC) and descending (DSC) tracks and GNSS observations carried
              out by ISIGN software, developed at our Institute based on \cite{Banyai2016}:
        \bitz
            \item identification of reflectors in interferograms and extraction LOS time-series,
            \item manual unwrapping of unwrapped phases if necessary,
            \item interpolation of ASC and DSC LOS time-series,
            \item fusion of LOS time-series with GNSS data using Kalman-filtering \cite{kalmanfilt}.
        \eitz
    \eitz
    \vspace{25pt}
    
    \bmini{0.475\textwidth}
        \finc{hungary_networks.png}
        \fcap{Integrated Artificial Reflector Benchmark networks in Hungary.}
    \emini
    \bmini{0.475\textwidth}
        \inc{daisy_schematic.png}
        \fcap{Geometric properties of satellite LOS directions. \cite{Banyai2016}}
    \emini

\eblock
\vspace{15pt}

\begin{ack}
    \small
    Contains modified Copernicus Sentinel data 2014-2018, processed by ESA.
\end{ack}
\vspace{15pt}

\begin{bib}{bcol}
    \tiny\bibliography{/home/istvan/Dokumentumok/texfiles/insar}
    \bibliographystyle{apalike}
\end{bib}


\newcommand{\kcap}[1] {
    \fcap{3D displacement time-series of combined \\InSAR LOS time-series and GNSS measurements at the #1-network.}
}


\bblock{Results}{bcol}
    \bfig{0.45}
        \inc{dszekcso_kalman.pdf}
        \kcap{Dunaszekcső}
    \efig
    \bfig{0.45}
        \inc{kulcs_kalman.pdf}
        \kcap{Kulcs}
    \efig
    \\[25pt]
    \bfig{0.45}
        \inc{fonyod_kalman.pdf}
        \kcap{Fonyód}
    \efig
    \bfig{0.45}
        \finc{dszekcso_refl_1.jpg}
        \fcap{Integrated Reflector Benchmark on top of a loess bank of Danube (near Dunaszekcső settlement).}
    \efig
\eblock


\bblock{Discussion and future plans}{bcol}
    \bitz
        \item The theory developed for the fusion of InSAR and GNSS datasets have been implemented and tested.
        \item GNSS measurments are needed for proper 3D displecment calculation (High
              deformation velocity of reflector benchmark).
        \item Reprocessing of interferograms with the IPTA Software \cite{ipta} (point target analysis,
              unwrapping). 
        \item Further development of ISIGN software: atmospheric correction algorithm for distant reflectors,
              interpolation of GNSS data to the position of coherent natural scatterers,
              rework \texttt{C} code in \texttt{C++} and integrate with Python.
        \item Process newly installed benchmark networks estabilished at Praid and Ciomadul-volcano, Transylvania. 
    \eitz
\eblock

\end{multicols}

\end{document}
