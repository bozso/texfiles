\documentclass{article}
\usepackage[a4paper, left=2cm, top=2cm, right=2cm, nohead]{geometry}
\usepackage[english]{babel}
\usepackage[utf8]{inputenc}
\usepackage[T1]{fontenc}

\begin{document}

\begin{center}
{\large \textsc{Benchmark of C-band radar corner reflectors based on Sentinel-1 SAR images. First results in the monitoring of the Dunaszekcső landslide (Hungary) using corner reflectors.} }

\vspace{10pt}

\textsc{István Bozsó\textsuperscript{1}, Eszter Szűcs\textsuperscript{1}, László Bányai\textsuperscript{1}, Viktor Weszterom\textsuperscript{1}}

\vspace{5pt}

\textsc{\textsuperscript{1}MTA CSFK Geodetic and Geophysical Institute}

\end{center}

\vspace{10pt}

Synthetic Aperture Radar (SAR) based interferometry (InSAR) is a technique widely used for the detection and monitoring of the deformation of different surfaces that reflect the electromagnetic (EM) wave emitted by the SAR sensor. SAR imaging is a coherent technique, meaning that not only the amplitude but the phase of the reflected EM wave is captured by the sensor. Calculation of surface displacement is possible from the phase differences produced by subtracting the phase values of one SAR scene from another SAR scene.

Deformations of the Earth’s surface caused by volcanoes, earthquakes, landslides and other surface processes are studied by the geoscientists using a single interferogram created from a SAR image pair acquired by spaceborne sensors. Time-series analysis of multiple interferograms made the assessment of slow, long-term deformation  possible, however displacement time-series derived from the time-series analysis, are only reliable when there are sufficient number of pixels that provide ``stable'' phase values over a long period of time. In areas where there are not enough stable pixels, interferometric processing becomes challenging if not impossible. In such areas, artificial scattering objects, so-called corner reflectors, can be installed that provide stable phase values for deformation monitoring purposes.

In the framework of an ESA PECS project (project ID: 4000118850/16/NL/SC) the Department of Broadband Infocommunications and Electromagnetic Theory of the Budapest University of Technology and Economics in cooperation with the MTA CSFK Geodetic and Geophysical Institute, developed a twin corner reflector based on the sensor characteristics of the Sentinel-1 satellites, capable of providing stable phase information for geodynamic investigations. The reflector geometry and dimensions were optimized for deformation monitoring using numerical and analogue modelling.
Several reflector networks were deployed  in Hungary. Three monitoring networks are located in areas susceptible to landslide activity: Fonyód (landslide area at Lake Balaton), Dunaszekcső and Kulcs (landslides near the banks of the Danube). Another test network is installed in Sopron, in a non-deforming area.

In this contribution we present the first deformation time-series of one test network in Dunaszekcső, Hungary. To capture the high gradient of the deformation and avoid the undersampling of the signals, both Sentinel-1A and B SAR scenes were processed. Altogether 73 scenes were processed with the GAMMA software in a small base approach forming three shortest temporal baseline interferograms. Interferograms are formed at 1 look in azimuth and 5 looks in the range direction  which corresponds to a pixel size of 14 m by 19 m. Topographic phase was removed by utilizing the 3-arcsec SRTM DEM and precise orbit vectors. Time-series analysis was carried out with the Stanford Method for Persistent Scatterers (StaMPS) software which uses spatial correlation of interferometric phase to identify phase stable pixels.  A rapid subsidence signal of 2.1 cm / 12 days was revealed in the eastern part of the area, suggesting that landslide is still ongoing and the area is not stabilized. Line-of-sight deformation was compared to campaign GNSS measurements. The results show that the measurement techniques,  i.e. GNSS and InSAR, are in the agreement within the error bounds.

\end{document}
