\documentclass[aspectratio=169]{beamer}
\usepackage[english]{babel}
\usepackage[utf8]{inputenc}
\usepackage[T1]{fontenc}
\usepackage{graphicx, tikz}
\usepackage[framemethod=TikZ]{mdframed}

\mode<presentation>
{
    \usetheme{Boadilla}
    \usecolortheme{default}
    \usefonttheme{default}
    \setbeamertemplate{navigation symbols}{}
    \setbeamertemplate{caption}[numbered]
} 


\def\SubdLithoVelo{Subducted Lithospheric Slab Velocity Structure - George Helffrich}

\newcommand\sepcite[2] {
    \begin{frame}
        \begin{center}
            \Huge \color{blue!55!black}
            #1
            
            \cite{#2}
        \end{center}
    \end{frame}    
}

\newcommand\paper[2]{
    Title: #1 \\
    Author: #2
}


\graphicspath{{/home/istvan/insar/images/geokemia/}}

\title[Geophysics of Subducted Slab]{Geophysics of the Subducted Slab and Accretion Wedge}
\author{István Bozsó}
\institute[MTA CSFK GGI]{MTA CSFK Geodetic and Geophysical Institute}
\date{2018.11.11.}

\begin{document}

\begin{frame}
    \titlepage
\end{frame}

\begin{frame}{Shorthand terms}
    \begin{table}
        \begin{tabular}{c|c}
            velo. & velocity \\
            subd. & subduction \\
            discont. & discontinuity \\
            freq. & frequency \\
            minerl. & mineralogy
        \end{tabular}
    \end{table}
\end{frame}

\begin{frame}{\SubdLithoVelo}
    \paper{Subducted Lithospheric Slab Velocity Structure: Observations and Mineralogical Inferences}{George Helffrich}
    \vspace{10pt}
    
    Abstract
    \begin{itemize}
        \item seismic wave speed structure $\rar$ anomalies 4-10\% relative to the ambient mantle
        \item transformations are taking place in the slab $\lar$ slab/mantle velocity contrasts do not monotonically decrease as a thermal origin would predict
        \item slab velocity layering must exist to depths excess of 200 km
        \item hydrous metabasalt velo. contrasts appear large enough to explain low velo. layers in shallow slabs; metastable anhydrous gabbro $\rar$ eclogite reaction not a good model for changes in shallow slab minerl.?
    \end{itemize}
\end{frame}

\begin{frame}{\SubdLithoVelo}
    Introduction
    \begin{itemize}
        \item subd. slabs delineated by seismicity but influence seismic travel times
        \item info. carried by 3 different ways
        \begin{itemize}
            \item travel time of the wave (known path $rar$ average path velo. or velo. anomaly)
            \item secondary arrivals $\lar$ presence of a slab (P $\rar$ S and S $\rar$ P waves from rock elastic property discont.-s); size and location of discont. from seismogram
            \item freq. dependent effects (certain wavelengths will be enhanced)
        \end{itemize}
        \item seismological investigation $\rar$ velo. (anomaly) in some part of the slab or layer thickness
        \item minerl.-cal inferences only from comparison with computed velo.-s
    \end{itemize}
\end{frame}


\begin{frame}{\SubdLithoVelo}
    \begin{minipage}[c]{0.45\textwidth}
        \inc{subduction_schematic2.png}
    \end{minipage}
    \hspace{10pt}
    \begin{minipage}[c]{0.45\textwidth}
        (a) P reflector and ScSp geometries \\
        (b) PS conversion geometry
    \end{minipage}
\end{frame}

\begin{frame}{\SubdLithoVelo}
    Seismological observations
    
    \begin{itemize}
        \item all observations from stations above subd. zones
        \item seismic wave -- slab interactions:
        \begin{itemize}
            \item reflection from the face of the subd.-ed slab
            \item refraction at its surface
            \item lengthy along-strike up-dip paths taken in the slab (slab = waveguide)
        \end{itemize}
        \item all of Japan overlies active subd. $\rar$ most of the records from here
        \item slab fave reflection (rarest) constrain material property contrast of mantle and subd. slab; here 4-10\% change in velo. in 10-20 km thick region
        \item S $\rar$ P, P $\rar$ S conversions (more common); geometry and P/S relative amplitude constrain conversion interface properties
    \end{itemize}
\end{frame}

\begin{frame}{\SubdLithoVelo}
    \begin{minipage}[c]{0.325\textwidth}
        \inc{japan_plate.png}
    \end{minipage}
    \hspace{10pt}
    \begin{minipage}[c]{0.6\textwidth}
        Sites where ScSp is observed, observed ScSp/ScS amplitude ratios. ScS $\rar$ P at $\approx$ 60 km depth (KMU, SHK), DDR conversion depth $\approx$ 180 km, MAT conversion depth $\approx$ 300 km.
    \end{minipage}
\end{frame}




\end{document}
