\sepcite{Kettős szeizmikus zóna}{abers}


\begin{frame}{Példa - Kettős szeizmikus zóna}
    \begin{figp}{\inc{abers_double_seism.png}}{}{0.45}{0.5}
        \begin{itemize}
            \item közepes mélységú földrengések a szubdukált kérgen kívül $\rar$ sok elméletet kizár
            \item lehetséges megoldás: második zóna $\rar$ rendellenes összetételű térrész a szubdukált köpeny litoszférán belül
            \item sok helyen észleltek kettős szeizmikus zónákat (double seismic zone - DSZ), számos szempontból hasonlóak
            \item két sáv 150-200 km-en összeér; tényleges effektus vagy artifakt? (mélységgel csökken a hipocentrum meghatározás pontossága)
        \end{itemize}
    \end{figp}
\end{frame}


\begin{frame}{Magyarázat}
    \begin{itemize}
        \item lemez görbülete $\rar$ feszültség; megfelelő megyarázat bizonyos területeken (nyugat-csendes-óceáni szubdukciós zónák)
        \item közepes mélységű földrengések fészekmechanizusai nagy ``szórás'' $\leftrightarrow$ homogén feszültségtér
        \item bazalt $\rar$ eklogit átalakulás és dehidratáció az óceáni lemezben $\leftrightarrow$ földrengések 20-40 km-el mélyebben a szubdukált köpeny litoszférában
    \end{itemize}
\end{frame}

\begin{frame}{Magyarázat}
    \begin{center}
    \begin{minipage}[c]{0.88\textwidth}
        \inc{abers_inhom_trap.png}
    \end{minipage}
    %\vspace{10pt}
    
    \begin{itemize}
        \item óceáni hátságok $\rar$ felemelkedő komplex olvadék $\rar$ litoszféra inhomogenitás; pl. mafikus magma felhalmozódása és megszilárdulása a kéreg alatt
        \item olvadék felhalmozódása nagy porozitású térrészekben
        \item földrengés? $\lar$ gázokban gazdag olvadék?
        \item nagy bizonytalanság; közvetlen mérési bizonyíték még nincs
    \end{itemize}
    \end{center}
\end{frame}
