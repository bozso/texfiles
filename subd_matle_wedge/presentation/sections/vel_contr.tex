\sepcite{Sebességkontraszt a szubdukálódó lemezben}{helffrich}

\begin{frame}{Példa - Sebességkontraszt a szubdukálódó lemezben}
    \begin{figp}{\inc{helf_japan_subd.png}}{}{0.35}{0.575}
        S $\rar$ P és P $\rar$ S fázisok; terjedési geometria és amplitúdó arányok $\rar$ konverziós réteg tulajdonságaira leírás;
        \vspace{5pt}
        
        \textbf{Pl.} alacsony sebességű réteg Észak-Honshu (Japán) alatt; 7 km vastagság, köpenyhez képest -6\%, lemezhez képest -12\%
        \vspace{20pt}
        
        ScSp fázis alapján $\rar$ helyi sebességkontraszt;
        \vspace{5pt}
        
        \textbf{Pl.} Tonga (Új-Zéland) 8-10 km vastag réteg a lemezben;
        
        \textbf{Pl.} Alaszka 2-10 km vastag réteg
        \vspace{10pt}
        
        Tonga lemez gyorsabb terjedési sebesség a köpenyhez képest, Japán és Alaszka lassabb; minden esetben a réteg $\le$ 10 km.
    \end{figp}
\end{frame}


\begin{frame}{Lemez - köpeny kontraszt}
    \textbf{alábukott lemez $\leftrightarrow$ köpeny (hőmérséklet, ásványtani összetétel):}
    \begin{itemize}
        \item hőmérsékletgradiens felszíntre merőlegesen változik leginkább, 400$^\circ$C 10 km-re a lemez felszínétől
        \item ásványtani ``kontraszt''
        \begin{itemize}
            \item lemez: olivin (ol), ortopiroxén (opx), klinopiroxén (cpx), plagioklász (pl), spinel (sp), gránát (gt)
            \item köpeny: gránát vagy spinel lherzolit
        \end{itemize}
        \item szeizmikus hullámsebességszámítások $\rar$ kízárólag a hőmérséklet- és az ásványtani kontraszt nem okozhatja az észlelt sebességkontrasztokat
    \end{itemize}
\end{frame}


\begin{frame}{Gabbró-eklogit átalakulás}
    \begin{itemize}
        \item ásványtani átalakulások $\rar$ sebességváltozás, sűrűségvaltozás a kéregben
        \item gabbró-eklogit, legsekélyebb átalakulás, +15\% sűrűségnövekedés
        \item \ce{(Mg,Fe)_2 Si O_4 (olivin) + Ca Al_2 Si_3 O_8 (anorthite) -> Ca (Mg,Fe)_2 Al_2 Si_4 O_12}
        \item felhajtőerő változás $\rar$ lefelé ``húzza'' a lemezt, földrengések a lemezben
        \item sekély átalakulás mélység (20 km)
        \item megfelelő model? ``száraz'' környezet feltételezés $\leftrightarrow$ fúrásminták hidratáltak
    \end{itemize}
\end{frame}


\begin{frame}{Szeizmikus sebesség metabazaltokban}
    \begin{minipage}[c]{0.525\textwidth}
        \centering
        \inc{helf_metabas_velo.png}
    \end{minipage}
    \hspace{10pt}
    \begin{minipage}[c]{0.4\textwidth}
        \begin{itemize}
            \item Metabazalt ásványok P sebességeinek kiszámítása $\lar$ hőmérséklet modellből
            \item kontraszt lemezfáciesekben képes sebességnövekedést előidézni a lemezrétegekben sekély alábukási szinteken
            \item észlelt sekély szeizmikus sebesség kontraszt konzisztens a számítot metabazalt sebességekkontraszttal
        \end{itemize}    
    \end{minipage}
\end{frame}
